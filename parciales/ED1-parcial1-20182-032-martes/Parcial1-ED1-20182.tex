\documentclass[10 pt]{article}
\usepackage{tikz}
\usetikzlibrary{arrows}
\usepackage[margin=0.5 in]{geometry}
\usepackage[utf8]{inputenc}
\usepackage{tabu}
\usepackage{color}
\usepackage{xcolor}
\usepackage{listings}
\usepackage{enumitem}
\usepackage{multicol}
\setlength{\columnsep}{1cm} 
\title{\textbf {Estructuras de Datos 1 - ST0245\\Examen Parcial 1
}}
\author{Nombre ..............................\\
		Departamento de Informática y Sistemas\\
		Universidad EAFIT\\}
\date{Septiembre 4 de 2018}
\begin{document}
\lstdefinestyle{customc}{
	language=Java, 
	numbers=left, 
	showspaces=false,
    showstringspaces=false, 
    tabsize=2, 
    breaklines=true,
    xleftmargin=5.0ex,
}
\lstset{escapechar=@,style=customc, numbers=left, stepnumber = 1} 
\maketitle
\begin{multicols}{2}
\section{Recursión 20\%}
\begin{enumerate}[label=(\alph*)]
	\item (10\%) Considere el siguiente programa. ¿Cuál es la salida generada por $fun(11, 5)$? Como un ejemplo: $fun(10, 3) = 20$.
	\begin{lstlisting}
	int fun(int n, int m){
	  if(n % m == 2) return n;
	  return fun(n + m, n - m);
	}
	\end{lstlisting}
	\begin{enumerate}[label=(\roman*)]
		%Respuesta: 22
		\item 11
		\item 5
		\item 22
		\item 2
	\end{enumerate}
	\item Considere el siguiente programa. ¿Cuál es la salida para $fun(1, 4)$? Como un ejemplo: $fun(1,2)=4$.
	\begin{lstlisting}
	int fun(int m,int n){
	  if(m==0){
	    return (n+1);  
	  }
	  if(m>0 && n==0){
	    return fun(m-1,1);  
	  }
	  int a=fun(m,n-1);
	  return fun(m-1,a);
	}
	\end{lstlisting}	
	\begin{enumerate}[label=(\roman*)]
		%Respuesta: 6
		\item 4
		\item 6
		\item 5
		\item 12
	\end{enumerate}
\end{enumerate}
\section{Complejidad 40\%}
\begin{enumerate}[label=(\alph*)]
\item (10\%) ¿Es $O(2^{k + 1})=O(2^k)$?
\begin{enumerate}[label=(\roman*)]
\item Sí
\item No
%Respuesta: Si
\end{enumerate}
\item (10\%) Considere el siguiente código escrito en Java. Encuentre la ecuación de recurrencia que mejor representa la complejidad asintótica en el peor de los casos.
\begin{lstlisting}
int f(int n){
  if(n <= 0){
    return 1;
  }
  int a = f(n / 2);
  int b = f(n / 2);
  int res = 0;
  for(int i = 0; i < n; i++){
    res += (a*b);
  }
  return res;
}
\end{lstlisting}
\begin{enumerate}[label=(\roman*)]
	%Respuesta: T(n) = 2T(n/2) + n
	\item $T(n) = 2T(n-1) + n$
	\item $T(n) = 2T(\frac{n}{2}) + n^2$
	\item $T(n) = 2T(\frac{n}{2}) + n$
	\item $T(n) = 2T(n-1) + (n-1)$
\end{enumerate}
\item (10\%) ¿Cuál de las siguientes afirmaciones es correcta para la función $func2(n, m)$?
\begin{lstlisting}
void func2(int n, int m){
  for(int i = 0; i < n; i++){
    for(int j = 0; j < m; j++){
      print(i,j);   
    }  
  }
}
\end{lstlisting}
\begin{enumerate}[label=(\roman*)]
%Respuesta O(n. m) pasos
\item Ejecuta $O(n + m)$ instrucciones
\item Ejecuta $O(n . m)$ instrucciones
\item Ejecuta $O(n^2 + m)$ instrucciones
\item Ejecuta $O(m + n^2)$ instrucciones
\end{enumerate}
\pagebreak
\item (10\%) Encuentre la ecuación de recurrencia que mejor representa la complejidad asintótica, para el peor de los casos, del siguiente código escrito en Java.
\begin{lstlisting}
int f(int n){
  if(n <= 0){
    return 1;
  }
  int new_n = n / 5;
  int re1 = f(new_n);
  int re2 = f(new_n);
  for(int i = 0; i < 4; ++i){
    re1 = re1 + f(n / 7);
  }
  return re1 + re2;
}
\end{lstlisting}
\begin{enumerate}[label=(\roman*)]
%Respuesta: a
\item $T(n) = 2 \times T(\frac{n}{5}) + 4 \times T(\frac{n}{7}) + C$
\item $T(n) = C \times T(\frac{n}{5}) + T(\frac{n}{7})$
\item $T(n) = C + 4 \times T(\frac{n}{5})$
\item $T(n) = C + 4 \times T(\frac{n}{7})$
\end{enumerate}
\end{enumerate}
\section{Notación O 20\%}
\begin{enumerate}[label=\alph*]
\item (10\%) Si $f(n) = O(n^2)$ y $g(n) = O(2^n)$, deduzca cuál es el valor de $O(f(n) \times g(n))$.
\begin{enumerate}[label=(\roman*)]
% Respuesta c : 2^n por n^2
\item $O(2^n)$
\item $O(n^2)$
\item $O(2^n \times n^2)$
\item $O(2^n + n^2)$
\end{enumerate}

\item (10\%) Sea $T(n) = 0.003 \times \log_{2} n + \log_{2}(\log_{2} n)$. Calcule $O(T(n)).$
\begin{enumerate}[label=(\roman*)]
\item $O(\log_{2}(\log_{2} n))$
\item $O(n ^ 2)$
\item $O(\log_{2} n)$
\item $O(n \log_{2} n)$
%Respuesta: O(log2 n)
\end{enumerate}
\end{enumerate}
\section{ArrayList 20\%}
\textbf{Nota: } El método \textbf{add(n)} añade el elemento $n$ en la última posición de la lista. El método \textbf{contains(n)} retorna verdadero si $n$ está en la lista, sino retorna falso. El método \textbf{size()} retorna el tamaño de la lista. El método \textbf{get(i)} devuelve el elemento en la posición $i$ de la lista. 
\begin{enumerate}[label=\alph*]
\item (10\%)¿Cuál es la complejidad asintótica, en el peor de los casos, de la siguiente función?
\begin{lstlisting}
void fun1(ArrayList<Integer> list){
  int a = -1, n=list.size();
  for(int i=n; i >= 0;i--){
  	a = Math.max(a, list.get(i));
  }
  for(int i = 0; i < n; i++){
  	for(int j = 0; j < a; j++){
  		list.add(i*j);
  	}
  }
}
\end{lstlisting}

\begin{enumerate}[label=(\roman*)]
\item $O(n^2)$
\item $O(\max(list) \times n)$
\item $O(\max(list) \times n^2)$
\item $O(n)$
% Respuesta: O(max{e} * n^2)
\end{enumerate}
\textbf{Nota}: La función $max(list)$ es el elemento mayor de la lista.
\item (10\%) ¿Cuál es la complejidad asintóntica, en el peor de los casos, de insertar
un elemento al principio de una lista hecha con arreglos (ArrayList)?
\begin{enumerate}[label=(\roman*)]
\item $O(n^2)$
\item $O(n )$
\item $O(log n)$
\item $O(1)$
%Respuesta: O(n)
\end{enumerate}
\end{enumerate}
\end{multicols}
\end{document}